\documentclass{scrartcl}

\title{Do we have any good reasons to trust our moral intuition?}
\author{\today}
\date{}

\usepackage[T1]{fontenc}
\usepackage{iftex}
\iftutex
        \usepackage{fontspec}
        \setmainfont{TeX Gyre Schola}
        \setsansfont{TeX Gyre Heros}
        \setmonofont{TeX Gyre Cursor}
        \usepackage{unicode-math}
        \setmathfont{TeX Gyre Schola}
\else
        \usepackage{tgschola}
        \usepackage{tgheros}
        \usepackage{tgcursor}
\fi
\usepackage[style=apa]{biblatex}
\addbibresource{mo.bib}
\usepackage{microtype}

\begin{document}
\maketitle

It cannot be denied, however, that intuitive judgment is occasionally
unreliable. But these are typically attributed to a lack of information.
For those who raise the example of a person whose moral principles are
considered ``utterly deranged'', it must also be noted how the
``expected'', or so called ``normative'' moral judgments as expected by
the people making this judgment, are no more fundamentally correct than
any other set of moral judgments, including those that are considered
``socially unacceptable''. A universal moral judgment for one moral act
among any set of different subjects cannot exist in a world that has
consistently overhauled its moral framework throughout the centuries,
and there is no evidence that we have reached any sort of stable
consensus on what moral acts are.

But if moral principles are up to each individual to decide, would there
be no consistency in society and no moral foundation for criminal
justice systems? My response would be that ``[the court's]
obligation is to define the liberty of all, not to mandate our own moral
code.'' \autocite[]{casey}.

\begin{itemize}
\item
  Compare so-called ``better'' sources of intuition, perhaps even
  empiricism. Consider the individuality.
\item
  Perhaps an extended example on the trolley problem.
\end{itemize}

\begin{center}\rule{0.5\linewidth}{0.5pt}\end{center}

Let us first make a distinction between two notions of trusting moral
intuition. One is that moral intuition is self-evident and that any
``sane'' human would arrive at a set of common moral judgments or
principles. Another is the notion that moral intuition is idiosyncratic
to each individual. I shall hereby assume that the latter is true, for
throughout history there has not been consistent agreement over what is
moral or what is ethical. Moral intuition doesn't necessarily align
across time periods: prior to the mid-20th century, homosexuality was
considered a sin in almost all jurisdictions and cultures often with
explanations that attempt to argue that homosexuality is ``obviously''
wrong---an appeal to intuition---yet we now condemn such discrimination;
neither does it align across cultural and geographical regions:
capitalism and individualism are predominant in some cultures while
socialism and collectivism are predominant in some others. (These are
indeed considered political ideologies, but nevertheless they are often
enforced through moral coercion and enjoy similar arguments on how they
are intuitively obvious.)

\emph{Discuss ``basic'' moral intuition like ``don't kill people'' span
across space and time?}

When we consider whether we should trust something, we must first
declare \emph{for what purpose} is this trust being contemplated. I
trust my general practitioner to treat my bronchitis, but I don't trust
them to give me accurate legal advice. This might seem like an obvious
example at first, but as we shall see, there are quite a few purposes
that people might apply moral intuition to, and the usage thereof are
not equally valid across these purposes.

There are usually four ways people use moral intuition. Firstly, it's to
make a moral judgment themselves: Is this specific event, in its
specific context, ``right or wrong''? Secondly, it's to decide on what
to do. Thirdly, it's to test a general moral principle: Does this
pattern sound right? Does applying this principle yield obviously
ridiculous results? Fourthly, it's to provide an interface to ethics.

I hereby argue that moral intuition is generally trustworthy in the
first and third usages, but is less effective in the second and fourth.

\section{Testing principles}

The common process of developing or considering the validity of a
ethical principle often consists of testing moral dilemmas based on
intuition, and, in doing so, refining the principle or rejecting it when
it yields a result or reasoning that we consider ``absurd''. For
example, I reject Kant's absolute prohibition against lying, as I
intuitively find it absurd to forbid myself from lying to a murderer who
threatens to find and kill my friend; I initially reject dogmatically
applying Bentham's utilitarianism due to intuitive counter-examples,
such as the intuitive immorality of pushing a heavy person off a bridge
to stop a runaway trolley (as describe in \autocite[21]{justice}).

So-called ``moral principles'' are only an generalized abstraction of
our moral judgment on individual events in their particular contexts,
i.e., ``it is wrong to kill \emph{this particular person in this
particular condition}'' is a moral judgment while ``it is wrong to kill
people'' is not. And it is in the moral judgment of these individual
events that intuition ultimately prevails. \footnote{I think this leads
  to a generalized argument on how the evaluation of every argument (or,
  every individual-case judgment, regardless of whether it is related to
  morality) boils down to intuition. Let's see how I could explain
  this\ldots{}}

\section{Ethics and morality}

In this context, what distinguishes ethics from morality is that ethics
are values that are culturally and collectively upheld, while morality
stems from an individual's sense of what is ``good'' and what is
``bad''. The process of the formulation of ethics could vary, although
they generally carry the goal of the preservation of a collective group
such as to form the basis of social order.

Intuition is naturally idiosyncratic even within one community expected
to possess similar ethics, which is sufficient to let us conclude that
intuition is not trustworthy for ethics.

But if morality is the moral judgment of an individual, can an
individual's moral intuition

\printbibliography

\end{document}
