\documentclass{scrartcl}

\title{Do we have any good reasons to trust our moral intuition?}
\author{\today}
\date{}

\usepackage[T1]{fontenc}
\usepackage{iftex}
\iftutex
        \usepackage{fontspec}
        \setmainfont{TeX Gyre Schola}
        \setsansfont{TeX Gyre Heros}
        \setmonofont{TeX Gyre Cursor}
        \usepackage{unicode-math}
        \setmathfont{TeX Gyre Schola}
\else
        \usepackage{tgschola}
        \usepackage{tgheros}
        \usepackage{tgcursor}
\fi
\usepackage[style=apa]{biblatex}
\addbibresource{mo.bib}
\usepackage{microtype}

\begin{document}
\maketitle

% Structure: I must define morality and ethics first, it seems.

Let us first make a distinction between two notions of trusting moral intuition. One is that moral intuition is self-evident and that any ``reasonable'' human would arrive at a set of common moral judgments or principles. Another is the notion that moral intuition is idiosyncratic to each individual.

There has not been consistent agreement over what is moral or what is ethical throughout history. Moral intuition does not completely align across time periods: prior to the mid-20th century, homosexuality was denounced in almost all jurisdictions and cultures often with explanations that attempt to argue that homosexuality is ``obviously'' wrong---an appeal to intuition---yet we now condemn such discrimination. Neither does moral intuition it align across cultural and geographical regions: capitalism and individualism are predominant in some regions of the world while socialism and collectivism are predominant in others; liberty is considered an inalienable principle in some but as a mere consequence of, or means to, efficiency in others. These are, of course, political ideologies; but they are often enforced through moral coercion and enjoy similar arguments on their "obviousness".

There seem to be, however, a set basic moral intuitions that seems inherent in the mind of each "civilized" person. Things such as "killing people is wrong, other than self-protection against perpetrators who already exhibit an extreme threat" may seem universally acknowledged. But they, in fact, are not: it is easy to dismiss the seeming exception of war and extremism, but the existence of "proudly killing enemies" and people who do not care about other's life at all (i.e. extremist groups) cannot be denied. And even if a generalized principle such as "murder is wrong" is universal, it does not naturally follow that each person yields an equivalent result for every individual case where a moral judgement is made, on whether killing at a particular moment with a well-defined context, is wrong.

Therefore, it is difficult, if not impossible, to define moral intuition based on the intuition of a collective group. I shall therefore proceed assuming that we refer to individual intuition.

% \emph{Discuss ``basic'' moral intuition like ``don't kill people'' span across space and time?}
\bigskip

When we consider whether we should trust something, we must first declare \emph{for what purpose} is this trust being contemplated. There are usually four ways people use moral intuition. Firstly, it's to make a moral judgment themselves: Is this specific event, in its specific context, ``right or wrong''? Secondly, it's to decide on what to do. Thirdly, it's to test a general moral principle: Does this pattern sound right? Does applying this principle yield obviously ridiculous results? Fourthly, it's to provide an interface to ethics.

I hereby argue that moral intuition is generally trustworthy in the first and third usages, but is less effective in the second and fourth. % RESTRUCT

\section{Testing principles}

The common process of developing or considering the validity of a ethical principle often consists of testing moral dilemmas based on intuition, and, in doing so, refining the principle or rejecting it when it yields a result or reasoning that we consider ``absurd''. For example, I reject Kant's absolute prohibition against lying, as I intuitively find it absurd to forbid myself from lying to a murderer who threatens to find and kill my friend; I initially reject dogmatically applying Bentham's utilitarianism due to intuitive counter-examples, such as the intuitive immorality of pushing a heavy person off a bridge to stop a runaway trolley (as describe in \autocite[21]{justice}).

So-called ``moral principles'' are only an generalized abstraction of our moral judgment on individual events in their particular contexts, i.e., ``it is wrong to kill \emph{this particular person in this particular condition}'' is a moral judgment while ``it is wrong to kill people'' is not. And it is in the moral judgment of these individual events that intuition ultimately prevails.\footnote{I think this leads to a generalized argument on how the evaluation of every argument (or, every individual-case judgment, regardless of whether it is related to morality) boils down to intuition. Let's see how I could explain this\ldots{}}
 
\section{Ethics and morality}

In this context, what distinguishes ethics from morality is that ethics are values that are culturally and collectively upheld, while morality stems from an individual's sense of what is ``good'' and what is ``bad''. The process of the formulation of ethics could vary, although they generally carry the goal of the preservation of a collective group such as to form the basis of social order.

Intuition is naturally idiosyncratic even within one community expected to possess similar ethics, which is sufficient to let us conclude that intuition is not trustworthy for ethics.

But if morality is the moral judgment of an individual, can an individual's moral intuition

\printbibliography

\end{document}

% vim: linebreak tw=65535 spell
