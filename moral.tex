\documentclass{scrartcl}

\title{Do we have any good reasons to trust our moral intuition?}
\author{}
\date{}

\usepackage[T1]{fontenc}
\usepackage{iftex}
\iftutex
        \usepackage{fontspec}
        \setmainfont{TeX Gyre Schola}
        \setsansfont{TeX Gyre Heros}
        \setmonofont{TeX Gyre Cursor}
        \usepackage{unicode-math}
        \setmathfont{TeX Gyre Schola}
\else
        \usepackage{tgschola}
        \usepackage{tgheros}
        \usepackage{tgcursor}
\fi
\usepackage[style=apa]{biblatex}
\addbibresource{moral.bib}
\usepackage{microtype}

\begin{document}
\maketitle

% Structure: I must define morality and ethics first, it seems.

\section{ultraduction}

Let us first make a distinction between two notions of trusting moral intuition. One is that moral intuition is self-evident and that people would arrive at a set of common moral judgments or principles. Another is the notion that moral intuition is idiosyncratic to each individual.

There has not been consistent agreement over what is moral or what is ethical throughout history. Moral intuition does not persist completely across time periods: prior to the mid-20th century, homosexuality was denounced in almost all jurisdictions and cultures often with explanations that attempt to argue that homosexuality is ``obviously'' wrong---an appeal to intuition---yet we now condemn such discrimination. Neither does moral intuition align across cultural and geographical regions: capitalism and individualism are predominant in some regions of the world while socialism and collectivism are predominant in others; liberty is considered an inalienable principle in some, but as a mere means to collective efficiency in others. These are, of course, political ideologies; but they are often enforced through moral coercion and enjoy similar arguments on their ``obviousness''.

There seems to be, however, a set of basic moral intuitions that seems present in the mind of each ``civilized'' person. Moral patterns often explained with intuition such as ``killing people is wrong, other than self-protection against perpetrators who already exhibit an extreme threat'' may \emph{seem} universally acknowledged. But that is an illusion: it is easy to dismiss the seemingly exceptional case of war and extremism, but the existence of serious notion of ``proudly and fearlessly killing enemies'' as a single counter-example falsifies the hypothesis that the so-called principle aforementioned is universal. While an analysis is warranted on each individual principle that claims to be universal, I doubt that any such claim would be supported.% other examples?

Even if a generalized principle such as ``murder is wrong [...]'' is universal, it does not naturally follow that each person yields an equivalent result for every individual case where a moral judgement is made, on whether killing at a particular moment with a well-defined context, is wrong.

Therefore, it is difficult, if not impossible, to discuss moral intuition in terms of a collective intuition of a group. I shall proceed assuming that we refer to individual intuition in this essay.

% \emph{Discuss ``basic'' moral intuition like ``don't kill people'' span across space and time?}
\bigskip

When we consider whether we should trust something, we must first declare \emph{for what purpose} is this trust being contemplated. I will iterate through them and discuss each, in the following sections.

\section{Making a moral judgement}

The core of my argument is that individual moral judgements of specific events are ultimately made by intuition.

Many philosophers traditionally labelled as moral intuitionists such as Prichard believe that moral intuition is self-evident \autocite{prichard}. I have discussed my rejection of this view in the introduction and I dissent his implicit assumption that moral judgement is nearly universal. But he is right in saying that moral judgement needs no proof. Let us then take a look at some alternatives to individual moral intuitionism, and I shall then discuss why these are less consistent and less correct than moral intuitionism.

What if we stick to moral frameworks such as Bentham's utilitarianism, Mill's revision of utilitarianism, Kantian categorical imperatives?

I, as an individual, reject some parts of them and accept others. The common process of developing or considering the validity of an ethical principle often consists of testing moral dilemmas based on intuition, and, in doing so, refining the principle or rejecting it when it yields a result or reasoning that we consider ``absurd''. For example, I reject Kant's absolute prohibition against lying, as I intuitively find it absurd to forbid myself from lying to a murderer who threatens to find and kill my friend---I make the hypothesis and imagine potentially contributing to my friend's chance of death for the sake of not telling lies, and I assert, intuitively, that it is not worth it. I also reject dogmatically applying Bentham's utilitarianism due to intuitive counter-examples, such as the intuitive immorality of pushing a heavy person off a bridge to stop a runaway trolley (as described in \autocite[21]{justice}). Therefore intuition on individual cases is the basis that gives birth to moral principles and framework.

Never have I seen a moral framework or principle that has a meticulous proof or reasoning on why it is correct, from first principles. And this is precisely because these first principles do not exist without intuition. For example, \textcite[2]{kant-intuition} claims that Kant is a moral intuitionist as his reasoning for categorical imperatives cites the need to ``[feel] pleasure or [\dots] delight in the fulfillment of duty'' \autocite{gmm}. \emph{Any} judgement on a contextualized real event by an individual is ultimately moderated by intuition, notwithstanding the thought-process that the individual uses at surface level.

% These so-called ``moral principles'' are only generalized abstractions of our moral judgment. Moral judgements apply to individual events in their particular contexts, i.e., ``it is wrong to kill \emph{this particular person in this particular condition}'' is a moral judgment while ``it is wrong to kill people'' is not. I argue that it is in the moral judgment of these individual events that intuition ultimately prevails.

% {\Huge\bfseries intuition is final in everything}

% TODO: Prove that intuition is the final arbitrator of results.
% \footnote{I think this leads to a generalized argument on how the evaluation of every argument (or, every individual-case judgment, regardless of whether it is related to morality) boils down to intuition. Let's see how I could explain this\ldots{}}


\section{Ethics and morality}

What distinguishes ethics from morality is that ethics are values that are culturally and collectively upheld, while morality stems from an individual's sense of what is ``good'' and what is ``bad''. The process of the formulation of ethics could vary, although they generally carry the goal of the preservation of a collective group such as to form the basis of social order.

Intuition is naturally idiosyncratic even within one community expected to possess similar ethics, which contraindicates the possibility of intuition as a trustworthy source of ethics. The goal of ethics is not the same as that of moral judgement; ethics attempts to produce generalized principles in a society that can provide consensus, stability, and collective prosperity. That is not a goal that individual intuition can attain, and is better suited for analysis of evidence and social conventions.

\section{Conclusion}

Therefore, pertaining to the moral judgement of individual events, moral intuition is not simply trustworthy; it is \emph{the} method of judgement that oversees all other methods of moral judgement and the development of moral frameworks or principles. But if the purpose of using moral intuition is the analysis of ethics, then the answer is a clear \emph{no}.

\printbibliography

\end{document}

% vim: linebreak tw=65535 spell
