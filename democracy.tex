\documentclass{scrartcl}
\usepackage[T1]{fontenc}
\usepackage{geometry}
\geometry{a4paper}
% \geometry{verbose,tmargin=1cm,bmargin=1cm,lmargin=1cm,rmargin=1cm,headheight=1cm,headsep=1cm,footskip=1cm}
\synctex=-1

\usepackage{babel}
\usepackage[style=apa]{biblatex}
\addbibresource{dem.bib}
\begin{document}
\title{Is there such a thing as too much democracy?}
\maketitle

\subsection*{Introduction}

The digital age has hardly changed the outlook of voting booths, but
the route from the ballots to voting booths has mutated enormously.
Voters used to go through information flows screen-filtered by fact-checking
systems, and with respected newspaper editors and prestigious news
anchors as gatekeepers to keep out the most deliberate untruths and
inflaming voices. Today they are besieged by waves of fake news, radicalized
and radicalizing viewpoints and discourses unleashed by social media
and the convenience of cell phones. When a political outsider and
demagogue strode into the oval office and resisted the departure from
it by smearing the election system and inciting the Capitol Hill insurrection,
the world smelled the deja vu of elected autocrats like Hitler in
1933 and Hugo Chavez of Venezuela. UK fared no better when Brexit
mandated by a referendum took three Prime Ministers to deliver, accompanied
by a political turmoil of three and a half years. The surging populism,
identity politics, partisan politics, and political tsunamis accentuate
one question. 
\begin{quote}
Is there already an overdose of democracy that grabs the rein of the
country into the hands of inciting demagogues and misinformed, impulsive
and irresponsible online mobs? 
\end{quote}
If the answer is NO, democracy has not been too much yet in the status
quo, the second question is open.
\begin{quote}
Is there a theoretical possibility that there can be too much democracy? 
\end{quote}

\subsection*{Definition of Democracy}

What democracy is needs to be defined before the evaluation of its
excess. Definitions of democracy have been evolving throughout human
history. 
\begin{description}
\item [{Definition}] 1 Rule by the majority
\end{description}
The original form of democracy is rule by the majority. Aristotle
identified three major forms of constitutions: Kingship, Aristocracy,
and Polity (the deviant of which is Democracy) if the government may
consist of one person, a few, or a multitude respectively \autocite{arispol}.
Direct democracy of rule by the majority was practiced in Athens in
the form of citizen's assembly sized 6000, with at last 40 annual
conventions. 
\begin{description}
\item [{Definition}] 2 Elective and Representative Democracy
\end{description}
To accommodate ever increasing population and to overcome the inherent
defects of direct governing by the people, the egalitarianism regime
where people make decisions together evolved into representatives
governing the country through competitive election into legislature
and administration \autocite{capsocdem}. The rank and file participate
indirectly in making laws and public policies by casting their votes.
Grass root democracy is replaced by representative democracy. 
\begin{description}
\item [{Definition}] 3 Liberal democracy based on institutionalized constitution
\end{description}
Since the 1990s, there has been a new academic definition of democracy,
the Compound Democracy, also known as Liberal Democracy or Constitutional
Democracy \autocite{Bao2018}.The equation is representative democracy
coupled with the rule of law, which safeguards the legitimacy of political
procedures, freedom of expression and participation of common citizens,
and confines the power of administers within constitutional and legal
principles. 

\bigskip{}

The defining line between the second and the third definition of democracy
lies in the absence or presence of the rule of law. If election procedures
are abused, political opponents are eliminated (imprison ed, exiled,
or killed), the regime has become an autocracy despite its claim to
democracy. However, if public resentments are fermented and maneuvered
to defy constitution and the rule of law and hijack governmental polices,
the regime still takes the formality of election and representation
of the people, but falls short of Liberal Democracy. Therefore, to
address the topical question in alignment with the definition of Liberal
Democracy, it is possible that even if the symptoms of rampant populism
hijacking public policies seem to be an overdose of democracy, the
underlying ill is the deficiency of Liberal Democracy.

This research is a comparative analysis set within the paradigms of
Elective and Representative democracy and Liberal Democracy. 

\subsection*{Yardstick to measure how much is too much}

The core issue of this research is the yardstick to measure how much
democracy is too much. If liberal democracy is taken as ends by its
own justifications, there is an overdose of it if it is crowding out
other inherent values of public life. If liberal democracy is taken
as means to social prosperity, it is in excess if it overwhelms other
means to social prosperity and makes the target even more inaccessible
than in the absence or mitigated presence of it. 

\subsection*{Democracy as ends}

Democracy is taken as ends for its intrinsic value of citizen participation
and decision as social good. Its importance is parallel to freedom,
equality, economic prosperity, social cohesion and public order. The
bar to evaluate democratic excess is: as long as it allows other kinds
of social goods to flourish as well, rather than crowds out others
out of our collective life, there is no overdose of democracy. 

Global landscape of democracy shows that vast regions are still short
of democracy. Of the 12 countries in the Middle East, 10 have never
experienced democracy. Asia is home to the largest number of countries
(40 per cent of countries in the region) that have never experienced
democracy at any time in their history. Venezuela, Brazil, Argentina,
Chile and multiple other Latin American countries have experienced
different degrees of democratic backslide\autocite{IDEA}. Cambodia,
Thailand, Pakistan, Turkey and Russia have retreated from democracy
into authoritarian regimes. Therefore, in vast regions of the world,
democracy is a scarcity rather than excess. 

However, in the most democratic regions of the world, North America
and Europe, the rise of populist, extremist and anti-establishment
candidates or parties is eroding people's trust in the government
and the entire constitutional institution. Some regard populism as
malaise or an overdose of democracy, and caution against the pervasiveness
of it.

Here comes a crucial question of the differentiation of populism from
liberal democracy. Populism still goes through representative legislature
and administration through competitive election (Definition 2), but
it goes out of line with Definition 3 of democracy (Liberal democracy
based on institutionalized constitution). Populist actors often show
disrespect for the accountability institutions that check government,
protect political pluralism and constitute liberal democracy. Public
opinions malnourished by misinformation and lack of deliberation,
or intentionally manipulated by demagogues hijack election and government
polices against the guardrails of constitution and rule of law. Populism
may look like democracy pro forma, but actually gnaws at the bones
of liberal democracy. The inherent predisposition for unconstrained
power turns populism into a threat for democracy.

Therefore, within the paradigm of liberal democracy, the raging populism
in North America or Europe is manifestation of democratic fragility
and erosion, rather than democratic overdose. 

\subsection*{Democracy as means}

John Stuart Mill thinks that democracy is best suited to promote the
common good in two ways. It first plays an important epistemic role
in identifying the common good. Universal suffrage and political participation
provide the best assurance that the interests of the governed will
be properly appreciated by political decision-makers. Secondly, the
political participation also improves the moral capacity of citizens.
The governed cultivate their deliberate capacity to exchange reasons
with others and to take principled stands in political process.\autocite{Mill_2010} 

Democracy is commonly acknowledged today as means to achieving common
good as economic growth, economic equality, gender equality, freedom,
education, lower crime rate etc.. However, disproportionately huge
importance attached to it is likely to overwhelm other effective means
to achieving these targets and thus make the targets even harder to
access. The following part will analyze the inefficiency of democracy
as means to prosperity. 

\subsubsection*{Democracy is not sufficient in achieving prosperity}

Democracies are by and large more affluent, more equal and less corrupt
than non-democracies and hybrid regimes. Europe and Latin America
have the highest representation of women in parliament ''\autocite[6]{IDEA},
while the democratically starved Middle East literally imposes gender
apartheid. Civil Liberties are one of the best-performing aspects
of democracy including in Africa. 

However, ``in and of itself, democracy is not sufficient to guarantee
low levels of corruption: indeed, 25 per cent of democracies globally
suffer from high levels of corruption ''\autocite[3]{IDEA}. Latin
America has highest socioeconomic inequality and highest crime rate.
USA has seen intense working class's sense of victim hood and economic
anger since they were left behind in globalization and free capital
flow. The old beacon of hope has been more stratified and polarized
in the past decade than ever before. Democracy does not translate
into prosperity. 

\subsubsection*{Democracy is not the only means to prosperity}

Authoritarian regimes like Park Chung-hee's reign worked the Miracle
of Han River of Korea in the 1960s and 70s. China has leaped to the
second economic powerhouse through 40 years of spectacular economic
growth. Lee Kuan Yew, celebrated as the architect of modern Singapore,
transformed Singapore from a modest port into a global financial and
transportation center with enormous improvement of people's living
standards and education. 

One the other hand, when countries with weak administration or antagonized
racial or religious groups are thrust into democracy, they suffer
schism, social turmoils and even coup. For example, Haiti copied American
constitution and the entire political system, but ended up being one
of the poorest and most corrupted countries in the world with political
turmoils one after another. Cambodia’s multi-party election democracy
plunged into a de facto one-party authoritarianism in 2017 due to
its weak state capacity. Democracy was imposed but collapsed for absence
of competent government. Democracy neither flourishes in weak states,
nor leads to common prosperity. 

Edmund Burke claims that good government is more important, and indeed
more valued by the people, than self government \autocite[70]{scand}.
Effective state is a very important prerequisite to economic growth
and social prosperity. Its importance could outweigh that of election
and representative democracy. 

Besides state building, there are other factors essential for social
prosperity. ``Institutions were more than just formal rules; they
encompassed the shared understanding of appropriate behavior that
overlay them'' \autocite[213]{demdie}. Political party's forbearance
not to leverage its full power to cripple its political adversaries,
mutual respect and tolerance of political candidates are tacit procedural
principles that make the democratic institutions function. The media's
rigorous gate keeping system guarantees the public sufficient access
to information and fair analysis. The public's willingness to deliberation
and rational exchange of opinions keeps them from the trap of the
echo chamber and online violence. All these are unwritten norms that
orchestrate to make the social soil fertile for the growth of democracy
and prosperity. 

\bigskip{}


\subsection*{Conclusion}

If democracy is taken as end, it is not excessive either in global
landscape or in established democracies. Many regions in this world
are starving for democracy. In established democracies like North
America or Europe, the surging populism is not overdose of democracy
but erosion of the backbones of democracy. 

If democracy is taken as means, it does not warrant social prosperity,
not is it the only means to prosperity. For regimes with severe social
stratification and racial or religious conflicts, the importance of
democracy submits to that of a strong government to contain the situation
to the track of development and prosperity. Multiple cases in history
have proven that absent strong state power, the orgy of democracy
ends up in chaos and collapse. Therefore, there has been too much
democracy in weak regimes. 

The fertile soil of civil society, including media's integrity and
rigorous fact-checking system, the public's political literacy and
willingness to step out of their conflicts of interests to engage
in deliberation and exchange of opinion, and the political climate
of compliance with the unwritten norms of forbearance and tolerance
are also crucial to guardrail the resilience and robustness of democracy,
cohesion and social stability and prosperity. The equation to prosperity
is state building, constitutional democracy, civil society and unwritten
norms of forbearance, tolerance etc.. If the symmetry of the equation
is broken, when democracy is prioritized at the expense of other factors
and becomes the only game in town, it paves the way to abyss. Democratic
backsliding could begin at the ballot box. It is too early to say
the symmetry in North America and Europe has already been broken,
but there is a likelihood that it could be. 

\printbibliography[heading=bibintoc]

\end{document}
